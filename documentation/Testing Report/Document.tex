\documentclass[12pt]{article}

\usepackage{hyperref}
\usepackage{datetime}
\usepackage{graphicx}
\usepackage{minted}
\usepackage{caption}
\usepackage{amssymb}
\usepackage{float}
\usepackage[a4paper,width=150mm,top=25mm,bottom=25mm]{geometry}
\usepackage[utf8]{inputenc}

\hyphenpenalty=100000
\hypersetup{
    colorlinks=true,
    linkcolor=blue,
    filecolor=blue,      
    urlcolor=blue,
    unicode=false
}
\renewcommand*\contentsname{Tabelă de Conținut}
\renewcommand{\listtablename}{Listă de Tabele}
\renewcommand\listoflistingscaption{Listă de Exemple}
\renewcommand*\listfigurename{Listă de Imagini}
\captionsetup[table]{name=Tabel}
\renewcommand\listingscaption{Exemplu}
\renewcommand*\figurename{Imagine}
\usemintedstyle{bw}
\setcounter{tocdepth}{2}

\title{
    \includegraphics[width=3cm, height=3.5cm]{Academy Logo.png} \\
    \vspace{5mm}
    {Document pentru Testarea\\
    Soluției \textbf{Aranea}}
    \vspace{10mm}
}
\vspace{20mm}
\author{
    Apostolescu Ștefan \\
    Băjan Ionuț-Mihăiță \\
    Brumă Ionuț-Cosmin \\
    Iosif George-Andrei \\
    Stanciu Răzvan-Daniel \\
    \vspace{10mm}
}
\ddmmyyyydate

\begin{document}

\null
\nointerlineskip 
\vfill
\let\snewpage \newpage
\let\newpage \relax
\maketitle
\let \newpage \snewpage
\vfill 
\break

\newpage

\tableofcontents

\newpage

\listoflistings

\newpage

\section{Detalii despre Document}

\subsection{Scop}
Acest document are scopul de a prezenta rezultatele etapei de testare a soluției software Aranea. Aceasta a presupus verificarea funcționării corespunzătoare și a alinierii dintre funcționalitățile finale, cele reieșite în urma perioadei de implementare, și a cerințele stabilite inițial, în documentul "\textit{Specificarea Cerințelor Software}".

\subsection{Conținut}
Documentul este împărțit în două capitole. Primul prezintă modul în care testarea a fost realizată și un rezumat al întregului proces efectuat. Al doilea capitol oferă o detaliere a fiecărui test în parte.

\newpage

\section{Procesul de Testare}

\subsection{Metodologie}

Testarea a fost efectuată atât prin intermediul unor teste manuale, în care s-a interacționat cu soluția software în linie de comandă, cât și automat, cu ajutorul unor teste unitare create cu ajutorul JUnit.

Din punct de vedere al statusului posibil al testelor efectuate, acestea au fost notate cu:
\begin{itemize}
    \item rezultat corect: rezultatele obținute au coincis cu cele așteptate (știute ca fiind corecte); sau
    \item rezultat parțial corect: rezultatele obținute diferă foarte puțin de cele așteptate.
\end{itemize}

Unele teste au necesitat modificări ale codului sursă. Acestea sunt menționate în dreptul testului la care au intervenit.

\subsection{Rezumat}

\begin{itemize}
    \item 34 de teste, dintre care:
    \begin{itemize}
        \item cu rezultat corect: 33
        \item cu rezultat parțial corect: 1
        \item unitare: 30
        \item manuale: 4
    \end{itemize}
    \item 5 metode ce au necesitat modificări
\end{itemize}

\newpage

\section{Teste Efectuate}

\subsection{Cazuri de Utilizare}

\subsubsection{Executarea Crawling-ului și Generarea de Hărți}

\begin{itemize}
    \item tip: manual
    \item descriere: rularea comenzii \\ \mintinline{bash}{aranea crawl links/list1.txt configurations/valid.config}, verificarea manuală a fișierelor descărcate local și a hărții generate
    \item precondiții: adăugarea alias-ului specific sistemului de operare
    \item fișiere de intrare: \mintinline{bash}{links/list1.txt} și \mintinline{bash}{configurations/valid.config}
    \item fișiere de ieșire: folder-ul \mintinline{bash}{downloads/127.0.0.1/}, populat cu fișiere
    \item status: rezultat parțial corect, din cauza inexistenței fișierului \mintinline{bash}{images/banner.jpg} ce este referențiat într-un fișier CSS
\end{itemize}

\subsubsection{Filtrarea Paginilor Salvate Local după o Anumită Extensie}

\begin{itemize}
    \item tip: manual
    \item descriere: rularea comenzii \\ \mintinline{bash}{aranea list jpg} și verificarea manuală a corectitudinii rezultatului printat în linie de comandă
    \item precondiții: adăugarea alias-ului specific sistemului de operare
    \item fișiere de intrare: fișiere specifice unui website descărcat local
    \item status: rezultat corect
\end{itemize}

\vspace{0.3cm}
\begin{listing}[ht]
    \begin{minted}{text}
	[..]
	127.0.0.1/images/pic01.jpg
	127.0.0.1/images/pic02.jpg
	127.0.0.1/images/pic03.jpg
	127.0.0.1/images/pic04.jpg
	127.0.0.1/images/pic05.jpg
	127.0.0.1/images/pic06.jpg
	[..]
    \end{minted}
    \caption{Extras din rezultatul filtrării paginilor salvate local după o anumită extensie}
    \label{listing:1}
\end{listing}
\vspace{0.3cm}

\subsubsection{Căutarea unui Șablon în Paginile Salvate Local}

\begin{itemize}
    \item tip: manual
    \item descriere: rularea comenzii \\ \mintinline{bash}{aranea search "Nunc Dignissim"} și verificarea manuală a corectitudinii rezultatului printat în linie de comandă
    \item precondiții: adăugarea alias-ului specific sistemului de operare
    \item fișiere de intrare: fișiere specifice unui website descărcat local
    \item status: rezultat corect
\end{itemize}

\vspace{0.3cm}
\begin{listing}[ht]
    \begin{minted}{text}
	[..]
	127.0.0.1/elements.html
	127.0.0.1/generic.html
	127.0.0.1/index.html
	[..]
    \end{minted}
    \caption{Extras din rezultatul căutării unui șablon în paginile salvate local}
    \label{listing:1}
\end{listing}
\vspace{0.3cm}

\subsubsection{Solicitarea Ajutorului}

\begin{itemize}
    \item tip: manual
    \item descriere: rularea comenzii \\ \mintinline{bash}{aranea help} și verificarea manuală a printării ghidului de utilizare a programului
    \item precondiții: adăugarea alias-ului specific sistemului de operare
    \item status: rezultat corect
\end{itemize}

\subsection{Clasa \mintinline{bash}{URLQueue}}

\subsubsection{Metoda \mintinline{bash}{getInstance}}

\begin{itemize}
    \item tip: unitar, în metoda \mintinline{bash}{URLQueueTest:getInstanceTest}
    \item descriere: verificarea obținerii unei instanțe
    \item precondiții: inițializarea listei
    \item status: rezultat corect
\end{itemize}

\subsubsection{Metoda \mintinline{bash}{remove}}

\begin{itemize}
    \item tip: unitar, în metoda \mintinline{bash}{URLQueueTest:removeElementFromEmptyList}
    \item descriere: verificarea extragerii unui element din listă când aceasta este goală
    \item precondiții: inițializarea listei
    \item modificări efectuate: verificarea listei goale
    \item status: rezultat corect
\end{itemize}

\subsubsection{Metoda \mintinline{bash}{remove}}

\begin{itemize}
    \item tip: unitar, în metoda \mintinline{bash}{URLQueueTest:removeMoreThanInserted}
    \item descriere: extragerea unui număr de elemente mai mare decât cel de elemente inserate
    \item precondiții: inițializarea listei și inserarea de elemente în listă
    \item status: rezultat corect
\end{itemize}

\subsection{Clasa \mintinline{bash}{Crawler}}

\subsubsection{Metoda \mintinline{bash}{getSavePath}}

\begin{itemize}
    \item tip: unitar, în metoda \mintinline{bash}{CrawlerTest:getSavePathWithoutBackslashSaveDir}
    \item descriere: verificarea comportamentului la un URL formatat greșit, care nu se termina în "\mintinline{bash}{/}"
    \item precondiții: inițializarea listei de URL-uri și a clasei testate, adăugarea de URL-uri în listă
    \item modificări efectuate: verificarea formatării corecte a șirurilor de caractere, formatarea lor dacă nu corespund
    \item status: rezultat corect
\end{itemize}

\subsubsection{Metoda \mintinline{bash}{getSavePath}}

\begin{itemize}
    \item tip: unitar, în metoda \mintinline{bash}{CrawlerTest:getSavePathMalformedURL}
    \item descriere: verificarea comportamentului la generarea căii de salvare cu URL-uri atipice
    \item precondiții: inițializarea listei de URL-uri și a clasei testate, adăugarea de URL-uri în listă
    \item fișiere de ieșire: directoare create
    \item status: rezultat corect
\end{itemize}

\subsubsection{Metoda \mintinline{bash}{downloadNextURL}}

\begin{itemize}
    \item tip: unitar, în metoda \mintinline{bash}{CrawlerTest:downloadNextUrlLocalBasic}
    \item descriere: descărcarea website-ului de pe \mintinline{bash}{localhost} cu un singur fir de execuție
    \item precondiții: inițializarea listei de URL-uri (cu lungime maximă de 20) și a clasei testate, adăugarea de URL-uri în listă
    \item fișiere de ieșire: surse ale website-ului descărcat
    \item status: rezultat corect
\end{itemize}

\subsubsection{Metoda \mintinline{bash}{downloadNextURL}}

\begin{itemize}
    \item tip: unitar, în metoda \mintinline{bash}{CrawlerTest:downloadNextUrlLocalMultiThreading}
    \item descriere: descărcarea website-ului de pe \mintinline{bash}{localhost} cu patru fire de execuție
    \item precondiții: inițializarea listei de URL-uri (cu lungime maximă de 20) și a clasei testate, adăugarea de URL-uri în listă
    \item fișiere de ieșire: surse ale website-ului descărcat
    \item modificări efectuate: moduri de inițializare, pornire și așteptare a firelor de execuție
    \item status: rezultat corect
\end{itemize}

\subsubsection{Metoda \mintinline{bash}{downloadNextURL}}

\begin{itemize}
    \item tip: unitar, în metoda \mintinline{bash}{CrawlerTest:downloadNextUrlPublicWebsiteBasic}
    \item descriere: descărcarea unui website de pe Internet cu un fir de execuție
    \item precondiții: inițializarea listei de URL-uri (cu lungime maximă de 120) și a clasei testate, adăugarea de URL-uri în listă
    \item fișiere de ieșire: surse ale website-ului descărcat
    \item status: rezultat corect
\end{itemize}

\subsubsection{Metoda \mintinline{bash}{downloadNextURL}}

\begin{itemize}
    \item tip: unitar, în metoda \\ \mintinline{bash}{CrawlerTest:downloadNextUrlPublicWebsiteMultiThreading}
    \item descriere: descărcarea unui website de pe Internet cu patru fire de execuție
    \item precondiții: inițializarea listei de URL-uri (cu lungime maximă de 120) și a clasei testate, adăugarea de URL-uri în listă
    \item fișiere de ieșire: surse ale website-ului descărcat
    \item modificări efectuate: modul de salvare a URL-urilor relative în pagina descărcată, inspectarea (și, opțional adăugarea automată a extensiei \mintinline{bash}{.html}) conținutului unei pagini fără extensie
    \item status: rezultat corect
\end{itemize}

\subsubsection{Metoda \mintinline{bash}{downloadNextURL}}

\begin{itemize}
    \item tip: unitar, în metoda \mintinline{bash}{CrawlerTest:downloadNextUrlSleep}
    \item descriere: verificarea funcționalității de întârziere a cererilor către website-uri
    \item precondiții: inițializarea listei de URL-uri și a clasei testate
    \item modificări efectuate: funcționalitatea de așteptare în cazul în care firul de execuție a extras cu succes un element din lista de URL-uri
    \item status: rezultat corect
\end{itemize}

\subsection{Clasa \mintinline{bash}{Sieve}}

\subsubsection{Metoda \mintinline{bash}{getInstance}}

\begin{itemize}
    \item tip: unitar, în metoda \mintinline{bash}{SieveTest:getInstance}
    \item descriere: verificarea preluării corectă a instanței clasei
    \item precondiții: inițializarea clasei și setarea caracteristicilor necesare
    \item status: rezultat corect
\end{itemize}

\subsubsection{Metoda \mintinline{bash}{addIgnoredPages}}

\begin{itemize}
    \item tip: unitar, în metoda \mintinline{bash}{SieveTest:addIgnoredPages}
    \item descriere: verificarea adăugării corecte a paginilor ignorate
    \item precondiții: inițializarea clasei și setarea caracteristicilor necesare
    \item status: rezultat corect
\end{itemize}

\subsubsection{Metoda \mintinline{bash}{checkURL}}

\begin{itemize}
    \item tip: unitar, în metoda \mintinline{bash}{SieveTest:checkURL}
    \item descriere: verificarea validării unui URL
    \item precondiții: inițializarea clasei și setarea caracteristicilor necesare
    \item status: rezultat corect
\end{itemize}

\subsubsection{Metoda \mintinline{bash}{checkContent}}

\begin{itemize}
    \item tip: unitar, în metoda \mintinline{bash}{SieveTest:checkContent}
    \item descriere: verificarea validării unui URL
    \item precondiții: inițializarea clasei și setarea caracteristicilor necesare (inclusiv a șablonului)
    \item fișiere de intrare: fișier local, pentru verificare
    \item status: rezultat corect
\end{itemize}

\subsubsection{Aruncarea unei Excepții de Tipul \mintinline{bash}{FailedRequestException}}

\begin{itemize}
    \item tip: unitar, în metoda \mintinline{bash}{SieveTest:testcheckURLexceptions}
    \item descriere: verificarea aruncării unei excepții în cazul unui URL invalid, care nu poate fi accesat
    \item precondiții: inițializarea clasei și setarea caracteristicilor necesare
    \item status: rezultat corect
\end{itemize}

\subsubsection{Aruncarea unei Excepții de Tipul \mintinline{bash}{FailedFileReadException}}

\begin{itemize}
    \item tip: unitar, în metoda \mintinline{bash}{SieveTest:testcheckContentexceptions}
    \item descriere: verificarea aruncării unei excepții în cazul unui fișier invalid
    \item precondiții: inițializarea clasei și setarea caracteristicilor necesare
    \item status: rezultat corect
\end{itemize}

\subsection{Clasa \mintinline{bash}{ExtensionFinder}}

\subsubsection{Constructor}

\begin{itemize}
    \item tip: unitar, în metoda \mintinline{bash}{ExtensionFinderTest:ExtensionFinder}
    \item descriere: verificarea atribuirii corecte a datelor
    \item precondiții: inițializarea clasei testate și setarea atributelor necesare
    \item status: rezultat corect
\end{itemize}

\subsubsection{Metoda \mintinline{bash}{parse}}

\begin{itemize}
    \item tip: unitar, în metoda \mintinline{bash}{ExtensionFinderTest:parse}
    \item descriere: verificarea găsirii tuturor fișierelor cu o anumită extensie
    \item precondiții: inițializarea clasei testate și setarea atributelor necesare
    \item fișiere de intrare: folder cu fișierele căutate
    \item status: rezultat corect
\end{itemize}

\newpage

\vspace{0.3cm}
\begin{listing}[ht]
    \begin{minted}{text}
    [..]
    D:/fdc@columbia.edu_files/default+en.css
    D:/fdc@columbia.edu_files/default.css
    D:/fdc@columbia.edu_files/fdc.css
    D:/fdc@columbia.edu_files/translateelement.css
    [..]
    \end{minted}
    \caption{Extras din rezultatul afișat în linie de comandă în urma căutării unor fișiere după extensie}
    \label{listing:1}
\end{listing}
\vspace{0.3cm}

\subsubsection{Metoda \mintinline{bash}{processFile}}

\begin{itemize}
    \item tip: unitar, în metoda \mintinline{bash}{ExtensionFinderTest:processFile}
    \item descriere: verificarea găsirii tuturor fișierelor cu o anumită extensie
    \item precondiții: inițializarea clasei testate și setarea atributelor necesare
    \item fișiere de intrare: folder cu fișierele căutate
    \item status: rezultat corect
\end{itemize}

\subsection{Clasa \mintinline{bash}{PatternFinder}}

\subsubsection{Constructor}

\begin{itemize}
    \item tip: unitar, în metoda \mintinline{bash}{PatternFinderTest:PatternFinder}
    \item descriere: verificarea atribuirii corecte a datelor
    \item precondiții: inițializarea clasei testate și setarea atributelor necesare
    \item status: rezultat corect
\end{itemize}

\subsubsection{Metoda \mintinline{bash}{parse}}

\begin{itemize}
    \item tip: unitar, în metoda \mintinline{bash}{PatternFinderTest:parse}
    \item descriere: verificarea găsirii tuturor fișierelor care potrivesc un anumit șablon
    \item precondiții: inițializarea clasei testate și setarea atributelor necesare
    \item fișiere de intrare: folder cu fișierele căutate
    \item status: rezultat corect
\end{itemize}

\vspace{0.3cm}
\begin{listing}[ht]
    \begin{minted}{text}
	[..]
	D:\fdc@columbia.edu_files\f.txt
	[..]
    \end{minted}
    \caption{Extras din rezultatul afișat în linie de comandă în urma căutării unor fișiere după un șablon}
    \label{listing:1}
\end{listing}
\vspace{0.3cm}

\subsection{Clasa \mintinline{bash}{SitemapGenerator}}

\subsubsection{Constructor}

\begin{itemize}
    \item tip: unitar, în metoda \mintinline{bash}{SitemapGeneratorTest:SitemapGenerator}
    \item descriere: verificarea atribuirii corecte a datelor
    \item precondiții: inițializarea clasei testate și setarea atributelor necesare
    \item status: rezultat corect
\end{itemize}

\subsubsection{Metoda \mintinline{bash}{parse}}

\begin{itemize}
    \item tip: unitar, în metoda \mintinline{bash}{SitemapGeneratorTest:parse}
    \item descriere: verificarea creării unei hărți specifice unei structuri de fișiere salvate local
    \item precondiții: inițializarea clasei testate
    \item fișiere de intrare: folder cu fișierele căutate
    \item status: rezultat corect
\end{itemize}

\subsubsection{Aruncarea unei Excepții de Tipul \mintinline{bash}{CannotWriteException}}

\begin{itemize}
    \item tip: unitar, în metoda \mintinline{bash}{SitemapGeneratorTest:parse}
    \item descriere: verificarea aruncării unei excepții 
    \item precondiții: inițializarea clasei testate
    \item fișiere de intrare: folder cu fișierele căutate
    \item status: rezultat corect
\end{itemize}

\subsection{Clasa \mintinline{bash}{AraneaLogger}}

\subsubsection{Metoda \mintinline{bash}{log}}

\begin{itemize}
    \item tip: unitar, în metoda \mintinline{bash}{AraneaLoggerTest:logAllToFile}
    \item descriere: verificarea creării unei fișier cu toate evenimentele jurnalizate
    \item fișiere de ieșire: \mintinline{bash}{out/test/aranea/com/aranea/all_logs.txt}
    \item status: rezultat corect
\end{itemize}

\subsubsection{Metoda \mintinline{bash}{log}}

\begin{itemize}
    \item tip: unitar, în metoda \mintinline{bash}{AraneaLoggerTest:filteredLogToFile}
    \item descriere: verificarea creării unei fișier numai cu acele evenimente care au o severitate mai gravă decât una minimă, setată
    \item fișiere de ieșire: \mintinline{bash}{out/test/aranea/com/aranea/filtered_log.txt}
    \item status: rezultat corect
\end{itemize}

\subsubsection{Aruncarea unei Excepții de Tipul \mintinline{bash}{LoggerOpenFileException}}

\begin{itemize}
    \item tip: unitar, în metoda \mintinline{bash}{SitemapGeneratorTest:parse}
    \item descriere: verificarea aruncării unei excepții în cazul în care calea către fișierul ce trebuie creat nu există
    \item status: rezultat corect
\end{itemize}

\subsection{Clasa \mintinline{bash}{AraneaConfiguration}}

\subsubsection{Constructor}

\begin{itemize}
    \item tip: unitar, în metoda \mintinline{bash}{AraneaConfigurationTest:parseValidConfig}
    \item descriere: verificarea inițializării corecte a clasei testate, prin compararea membrilor clasei cu valorile corecte
    \item fișiere de intrare: \mintinline{bash}{tests/com/aranea/files/configurations/valid.config}
    \item status: rezultat corect
\end{itemize}

\subsubsection{Aruncarea unei Excepții de Tipul \mintinline{bash}{ConfigurationOpenFileException}}

\begin{itemize}
    \item tip: unitar, în metoda \mintinline{bash}{AraneaConfigurationTest:detectNonexistentFile}
    \item descriere: verificarea aruncării unei excepții în cazul în care fișierul de configurare nu există
    \item status: rezultat corect
\end{itemize}

\subsubsection{Aruncarea unei Excepții de Tipul \mintinline{bash}{ConfigurationMissingKeysException}}

\begin{itemize}
    \item tip: unitar, în metoda \mintinline{bash}{AraneaConfigurationTest:detectInvalidConfig}
    \item descriere: verificarea aruncării unei excepții în cazul în care fișierul de configurare nu conține toate cheile obligatorii
    \item fișiere de intrare: \mintinline{bash}{tests/com/aranea/files/configurations/invalid.config}
    \item status: rezultat corect
\end{itemize}

\end{document}